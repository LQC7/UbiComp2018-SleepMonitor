<<<<<<< HEAD
\section{INTRODUCTION}\label{sec:1introduction}

Sleep plays a vital role in good health and well-being throughout a person's life. Lacking of sleep or having poor sleep can lead to
serious, sometimes life-threatening, health problems~\cite{lallukka2016contribution}. The quality of sleep depends on many factors,
including the sleeping environment such as the light intensity and the ambient temperature, the user's breath, the sleeping posture, and
the bedtime routine. By capturing and modeling these factors, sleep monitoring can help users to assess and understand their sleeping
pattern, and provide feedback on how the change of the sleeping environment and behavior affect the sleep quality.

Traditionally, sleep monitoring had to be done in a clinic environment using dedicated medical equipment.  Modern mobile devices such as
smartphones and smartwatches offer a new way for sleep monitoring. By utilizing the rich mobile sensors, we can track certain activities
such as body movements or snore, and to use the tracked information to detect and model sleep
patterns~\cite{zeo,Jawbone,SleepAndroid,fitbit,gu2016sleep}. Such an approach is known as
\emph{actigraphy}~\cite{Actigraphy,ancoli2003role}. An actigraphy approach has several advantages over a clinic solution, including being
lower cost and non-invasive, and can be performed at home on an on-going basis.

%like electroencephalography (EEG), \`{o}electrocardiography (ECG) and electromyography (EMG)

However, existing mobile-based sleep monitoring applications fail to fully exploit the sensors provided by modern mobile devices. As a
result, current mobile-based sleep monitoring system typically only provide superficial information such as the duration of deep and
shallow sleep. Without further detailed information of the environment and activities across sleep stages, these applications offer little
help in understanding the causes of poor sleep. To unlock the potential of mobile-based sleep monitoring, we need to find ways to take full
advantage of the rich sensor data provided by modern mobile devices.


The work presented by Gu \etal~\cite{gu2016sleep} is among the first attempts in this direction
\cite{lawson2013validating,bauer2012shuteye,min2014toss,al2014classifying}. Their work exploits smartphone sensors to detect physical and
acoustical activities happened during sleep. While promising, this approach requires placing the smartphone next to the user's head and
ensuring the phone remains stationary throughout the sleeping process. However, this requirement often cannot be satisfied and many users
do not want to place the mobile phone too close to the body due to health risk concerns~\cite{StepHealth,Quorasleep}.

\begin{figure}[!t]
\centering
\setlength{\belowcaptionskip}{-13pt}
      \includegraphics[width=0.5\textwidth]{Figures/datacollect.pdf}
  \caption{{\systemname} detects a wide range of sleep-related activities and events using a smartwatch.}\label{fig:datacollect}
\end{figure}

This paper presents a better approach. Instead of using smartphones and relying on strict, unrealistic assumptions, we exploit the rich
sensor data provided by smartwatches to track a wider range of sleep-related activities. The advantages of using a smartwatch are that
many users are willing to wear the device throughout the night, thus the device can remain relatively close to the user over the duration
of sleep.  The smartwatch also allows us to collect a richer set of information. The collected information can then be used to monitor a
wider range of sleep-related events. While there are prior works on sleep monitoring based on smartwathes
\cite{pombo2016ubisleep,shelgikar2016sleep,haescher2015anomaly,borazio2012combining}, these approaches only gather coarse-grained
information and often rely on specialize devices such as pressure mattresses or image acquisition equipment. In this work, we demonstrate
that, for the first time, smartwatches enable us to collect an extensive set of sleep-related events -- many of which are not supported in
prior work. Having a more comprehensive set of sleep-related events and activities will thus enable users to gain a deeper understanding of
their sleep patterns and the causes of poor sleep.



In this paper, we present \systemname, the first smartwatch-based monitoring system that can track a wide range of sleep-related
activities. \systemname gathers sleep-related activities by utilizing the commonly available sensors on smartwatches: the accelerometer,
gyroscope, microphone and ambient light sensor, etc. It then uses the tracked information to infer the user's sleep posture and habits --
thing like changes of body and hand positions, as well as sound events due to e.g. snoring or coughing.  Collecting these data can help a
user to gain a deep understanding of his/her sleeping pattern and quality, and to find ways to improve sleep.

However, translating the collected senor data to these sleeping events is non-trivial due to the changing nature of smartwatches during
sleep. To overcome these challenges, we develop a set of new methods and analysis, specifically targeting portable mobile devices.  The key
insight behinds {\systemname} is that the sleep quality is strongly correlated to the sleep posture, acoustic events and the illumination
condition \cite{shelgikar2016sleep}. By exploiting the unique characteristics of sensory data produced by different sleep activities, we
develop a series of novel algorithms to correlate the sensory data to sleep events. We then design a model to incorporate the detected
events to
infer the user's sleep stages and sleep quality.

We evaluate our approach by applying it to ten users over two weeks period. The experimental results show that {\systemname} is  effective
and accurate in capturing sleep-related activities. For body posture classification, body rollover recognition, hand position
identification and  acoustic events detection,  it achieves least accuracies of 98\%,  90\%,  87\% and  96.9\%, respectively. These results
show that {\systemname}  is superior to existing actigraphy-based applications.

Our main contributions are:

\begin{itemize}[itemsep=1mm,nolistsep]

\item We present a novel sleep monitoring system based on smartwatch sensor information. Our system captures a wider range of sleeping
    activities that none of the current mobile-based sleep monitoring solutions can offer.

\item We design a set of new algorithms and analysis to effectively exploit the smartwatch sensor data to detect sleep-related
    activities. These include body postures, body rollovers, hand positions, micro body movements and acoustical events.

\item The experimental results suggest that our prototype system, \systemname, is highly effective in capturing sleep-related events.

\end{itemize}
=======
\section{INTRODUCTION}\label{sec:1introduction}

Sleep plays a vital role in good health and well-being throughout a person's life. Lacking of sleep or having poor sleep can lead to
serious, sometimes life-threatening, health problems~\cite{altena2008sleep,chandola2010effect,lallukka2016contribution}. The quality of
sleep depends on many factors, including the sleeping environment such as the light intensity and the ambient temperature, the user's
breath, the sleeping posture, and the bedtime routine. By capturing and modeling these factors, sleep monitoring can help users to assess
and understand their sleeping pattern, and provide feedback on how the change of the sleeping environment and behavior affect the sleep
quality.

Traditionally, sleep monitoring had to be performed in a clinic environment using dedicated medical equipment.  Modern mobile devices such
as smartphones and wearable devices offer a new way for sleep monitoring. By utilizing the rich mobile sensors, we can track certain
activities such as body movements or snore, and to use the tracked information to detect and model sleep
patterns~\cite{zeo,Jawbone,SleepAndroid,fitbit,gu2016sleep}. Such an approach is known as
\emph{actigraphy}~\cite{Actigraphy,ancoli2003role}. An actigraphy approach has several advantages over a clinic solution, including being
lower cost and non-invasive, and can be performed at home on an on-going basis.

%like electroencephalography (EEG), \`{o}electrocardiography (ECG) and electromyography (EMG)

However, existing mobile-based sleep monitoring applications fail to fully exploit the rich sensors provided by modern mobile devices. As a
result, current mobile-based sleep monitoring system typically only provide coarse-grained data such as the sleep duration and body
movement. Without further detailed information of the environment and activities across sleep stages, these applications offer little help
in understanding the causes of poor sleep. To unlock the potential of mobile-based sleep monitoring, we need to find ways to take full
advantage of the rich sensor data provided by modern mobile devices.


The work presented by Gu \etal~\cite{gu2016sleep} is among the first attempts to gather  a wide range of sleep events using mobile sensors.
There are two main drawbacks of this approach. Firstly, it requires placing the smartphone next to the user's head and ensuring the phone
remains stationary throughout the sleeping process. This requirement often cannot be satisfied because (a) the phone is often moved during
sleep due to body movements and (b) many users do not want to place the mobile phone too close to their body due to health risk
concerns~\cite{StepHealth,Quorasleep}.  Secondly, their approach can only gather coarse-grained sleep data due to the way the mobile phone
is used. Recently, Sun \etal~\cite{sleepmonitor} show that one can exploit the smartwatch sensors to monitor the respiratory  rate and body
position. While promising, this work only captures two sleep events and thus only scratches the surface of what could be possible.


\begin{figure}[!t]
\centering
\setlength{\belowcaptionskip}{-13pt}
      \includegraphics[width=0.5\textwidth]{Figures/datacollect.pdf}
  \caption{Our approach detects a wide range of sleep-related activities and events using a smartwatch.}\label{fig:datacollect}
\end{figure}

This paper presents novel approach to gather a richer set of fine-grained sleep events. Instead of using smartphones and relying on strict,
unrealistic assumptions, we exploit the rich sensor data provided by smartwatches to track a wider range of sleep-related activities. The
advantages of using a smartwatch are that many users are willing to wear the device throughout the night, thus the device can remain
relatively close to the user over the duration of sleep. The smartwatch also allows us to collect a richer set of information. The
collected information can then be used to monitor a wider range of sleep-related events. While there are prior works on sleep monitoring
based on smartwathes \cite{pombo2016ubisleep,shelgikar2016sleep,haescher2015anomaly,borazio2012combining}, these approaches only gather
coarse-grained information and often rely on specialize devices such as pressure mattresses or image acquisition equipment. In this work,
we demonstrate that, for the first time, smartwatches enable us to collect an extensive set of sleep-related events -- many of which are
not supported in prior work. Having a more comprehensive set of sleep-related events and activities will thus enable users to gain a deeper
understanding of their sleep patterns and the causes of poor sleep.



In this paper, we present \systemname, the first smartwatch-based monitoring system that can track a wide range of sleep-related
activities. \systemname gathers sleep-related activities by utilizing the commonly available sensors on smartwatches: the accelerometer,
gyroscope, microphone and ambient light sensor, etc. It then uses the tracked information to infer the user's sleep posture and habits --
thing like changes of body and hand positions, as well as sound events due to e.g. snoring or coughing.  Collecting these data can help a
user to gain a deep understanding of his/her sleeping pattern and quality, and to find ways to improve sleep.

However, translating the collected senor data to these sleeping events is non-trivial due to the changing nature of smartwatches during
sleep. To overcome these challenges, we develop a set of new methods and analysis, specifically targeting portable mobile devices.  The key
insight behinds {\systemname} is that the sleep quality is strongly correlated to the sleep posture, acoustic events and the illumination
condition \cite{shelgikar2016sleep}. By exploiting the unique characteristics of sensory data produced by different sleep activities, we
develop a series of novel algorithms to correlate the sensory data to sleep events. We then design a model to incorporate the detected
events to
infer the user's sleep stages and sleep quality.

We evaluate our approach by applying it to ten users over two weeks period. The experimental results show that {\systemname} is  effective
and accurate in capturing sleep-related activities. For body posture classification, body rollover recognition, hand position
identification and  acoustic events detection,  it achieves least accuracies of 98\%,  90\%,  87\% and  96.9\%, respectively. These results
show that {\systemname}  is superior to existing actigraphy-based applications.

Our main contributions are:

\begin{itemize}[itemsep=1mm,nolistsep]

\item We present a novel sleep monitoring system based on smartwatch sensor information. Our system captures a wider range of sleeping
    activities that none of the current mobile-based sleep monitoring solutions can offer.

\item We design a set of new algorithms and analysis to effectively exploit the smartwatch sensor data to detect sleep-related
    activities. These include body postures, body rollovers, hand positions, micro body movements and acoustical events.

\item The experimental results suggest that our prototype system, \systemname, is highly effective in capturing sleep-related events.

\end{itemize}
>>>>>>> 8ff952559e42dc8fe674cbb4b2feae3334fa8483

\section{INTRODUCTION}\label{sec:1introduction}

Sleep plays a vital role in good health and personal well-being throughout one's life. Lack of sleep or poor quality of sleep can lead to
serious, sometimes life-threatening, health problems~\cite{altena2008sleep,chandola2010effect,lallukka2016contribution}, decrease level of cognitive performance~\cite{alhola07sleep,akerstedt07altered}, and affect mood and feelings of personal well-being~\cite{paunio09longitudinal,pilcher97sleep}.
Besides having an adverse effect to individuals, insufficient or poor quality sleep has a significant economic burden, among others, through decreased productivity, and medical and social costs associated with treatment of sleep disorders~\cite{hafner17why}. Indeed, to highlight the significance of sleep quality, the Centre for Disease Prevention (CDC) has declared insufficient sleep
as a public health problem in the US\footnote{\url{https://edition.cnn.com/2015/02/18/health/great-sleep-recession/index.html}}, and the concern is widely shared amongst other industrialized countries.

To improve quality of sleep, off-the-shelf solutions for \emph{sleep monitoring} have recently emerged as a popular way to obtain information about one's sleeping patterns~\cite{ko15consumer}. By taking advantage of diverse sensors, behaviours and routines associated with sleeping can be captured and modelled. This in turn can help users understand their sleep behaviour and provide feedback on how to improve their sleep, for example, by changing routines surrounding sleep activity or improving the sleeping environment. What makes self monitoring particularly attractive is the non-invasive nature of the sensing compared to the traditional, clinical approach which requires sleeping within a clinical environment and being equipped with dedicated sensing equipment, e.g., relying on EEG, ECG or EMG signals~\cite{ebrahimi2008automatic,saper2005hypothalamic,oropesa1999sleep,langkvist2012sleep}. Examples of consumer-grade sleep monitors range from apps running on smartphones or tablets to smartwatches and 
specialized wearable devices~\cite{zeo,Jawbone,SleepAndroid,fitbit,gu2016sleep,sleepmonitor}. 

%Sleep monitoring using consumer-grade devices is also known as \emph{actigraphy}~\cite{Actigraphy,ancoli2003role} as the key goal is to capture and characterize periods of inadct
%Traditionally, sleep monitoring had to be performed in a clinic environment using dedicated medical equipment.  Modern mobile devices such
%as smartphones and wearable devices offer a new way for sleep monitoring. By utilizing the rich mobile sensors, we can track certain
%activities such as body movements or snore, and to use the tracked information to detect and model sleep
%patterns. Such an approach is known as
%. Compared to a clinic solution, an actigraphy approach the advantages of being lower
%cost and non-invasive, and can be performed at home on an on-going basis.

%Current solutions to consumer grade sleep monitoring predominantly are capable of estimating overall sleep quality
Despite the popularity of consumer-grade sleep monitors, currently the full potential of these devices is not being tapped into. While current consumer-grade solutions can accurately estimate overall sleep quality, model extent of sleep states, and capture specific sleep disorders~\cite{kay2012lullaby,zhang2013real,sleepmonitor}, they offer little help in understanding the characteristics that surround poor sleep and are unable to provide recommendations on how to overcome these problems. This is because current solutions focus on monitoring characteristics of the sleep itself, without considering also behaviours occurring during sleep, and the environmental context affecting sleep, e.g., ambient light-level and noise. Indeed, sleep quality has been shown to depend on a wide range of factors with sleep environment and behaviours during and before sleep being some of the most important ones. For example, intensity of ambient light~\cite{hood04determinants}, temperature\textcolor{blue}{~\cite{}, and noisiness~\cite{}} of the environment can significantly affect sleep quality. Similarly, the user's breathing patterns, posture during sleep, and routines surrounding the bedtime also have a significant impact on sleep quality\textcolor{blue}{~\cite{}}. Without details of the environment and activities across sleep stages,  monitoring solutions are unable to offer help in understanding the characteristics of these factors, and the root cause of poor sleep cannot be captured. To fully unlock the potential of consumer-grade sleep monitoring, more powerful ways of taking advantage of the rich sensor data accessible through these devices are required.

%new ways to take full
%advantage of the rich sensor data provided by modern mobile devices.


%By capturing and modeling these factors, sleep monitoring can help users to assess
%and understand their sleeping pattern, and provide feedback on how to change the sleeping environment or routines associated with sleep.


%like electroencephalography (EEG), \`{o}electrocardiography (ECG) and electromyography (EMG)

%However, existing mobile-based sleep monitoring applications fail to fully exploit the rich sensors provided by modern mobile devices. As a
%result, current mobile-based sleep monitoring systems typically only provide coarse-grained information such as the sleep duration and body
%movement.


%The work presented by Gu \etal~\cite{gu2016sleep} is among the first attempts to gather  a wide range of sleep events using mobile sensors.
%There are two main drawbacks of this approach. Firstly, it requires placing the smartphone next to the user's head and ensuring the phone%
%%%remains stationary throughout the sleeping process. This requirement often cannot be satisfied because (a) the phone is often moved during
%sleep due to body movements and (b) many users do not want to place the mobile phone too close to their body due to health risk
%concerns~\cite{StepHealth,Quorasleep}.  Secondly, their approach can only gather coarse-grained sleep data due to the way the mobile phone
%is used. Recently, Sun \etal~\cite{sleepmonitor} show that one can exploit the smartwatch sensors to monitor the respiratory  rate and body
%position. While promising, this work only captures two sleep events and thus only scratches the surface of what could be possible.


%\begin{figure}[!t]
%\centering
%\setlength{\belowcaptionskip}{-13pt}
%      \includegraphics[width=0.5\textwidth]{Figures/datacollect.pdf}
%  \caption{Our approach detects a wide range of sleep-related activities and events using a smartwatch.}\label{fig:datacollect}
%\end{figure}

%advantages of using a smartwatch are that many users are willing to wear the device throughout the night, thus the device can remain
%relatively close to the user over the duration of sleep.


In this paper, we contribute by presenting the design and development of \systemname, a \emph{holistic sleep monitoring solution} that captures rich information about sleep events, the environmental context of sleeping, and the overall quality of sleep. \systemname exploits sensor modalities available on off-the-shelf smartwatches to capture a wider range of sleep-related activities than what is available in previous sleep monitoring solutions (see Table~\ref{tab:function}). The key insight
in {\systemname} is that the sleep quality is strongly correlated to the sleep posture, acoustic events and the illumination condition
\cite{shelgikar2016sleep}. By using a smartwatch,
the sensors are close to the user during all stages of the night, enabling detailed capture of not only sleep cycles, but body movements and environmental changes happening during the sleep period. Translating the collected sensor data to these sleeping events is non-trivial due to changes in sensor measurements caused by hand motions during sleep. To overcome this challenge, we develop a set of new methods for analysing and capturing sleep-related information from sensor measurements available on a smartwatch. By exploiting the unique characteristics of sensory data produced by different sleep activities, we develop a
series of novel algorithms to correlate the sensory data to sleep events. We then design a model to incorporate the detected events to
infer the user's sleep stages and sleep quality. While some prior research has examined the use of smartwatches for sleep monitoring~\cite{pombo2016ubisleep,shelgikar2016sleep,haescher2015anomaly,borazio2012combining}, these approaches have only been able to gather coarse-grained information about sleep and often required additional highly-specialized devices, such as pressure mattresses or image acquisition equipment. In this article, we demonstrate that, for the first time, using {\em only a smartwatch}, it is possible to capture an extensive set of sleep-related information -- many of
which are not presented in prior work. By having a more comprehensive set of sleep-related events and activities available will enable users to gain a deeper understanding of their sleep patterns and the causes of poor sleep, and to make recommendations on how to improve one's sleep quality. 



%We implement our approach in a prototype system called \systemname. It gathers sleep-related activities by utilizing the commonly available
%sensors on smartwatches: the accelerometer, gyroscope, microphone and ambient light sensor, etc. It then uses the tracked information to
%infer the user's sleep posture and habits -- thing like changes of body and hand positions, as well as sound events due to e.g. snoring or
%coughing.  Collecting these data can help a user to gain a deep understanding of his/her sleeping pattern and quality, and to find ways to
%improve sleep. \FIXME{ZW: The introduction is too wordy. It needs to get to the point quicker. I will get back to this later.}

  % \FIXME{ZW: This one needs to be merged with the previous paragraph.}
  
We evaluate \systemname through rigorous and extensive benchmark experiments conducted on data collected from fifteen participants during a two week monitoring period. The results of our experiments demonstrate that \systemname can accurately characterize body motions and movements during sleep, as well as capture different acoustic events. Specifically, the lowest accuracy for \systemname in our experiments is 87\%, with the best event detection accuracy reaching up to 98\%. We also evaluate the overall performance of \systemname, demonstrating that it can accurately detect various sleep stages, helping user to better understand their sleep quality.

The contributions of the paper are summarized as follows:
\begin{itemize}[noitemsep]
	\item We present the design and development of \systemname, a holistic sleep monitoring solution that relies {\em solely} on sensor data available on contemporary smartwatch to capture a wide range of sleep-related information.
	\item We develop novel algorithms for capturing sleep-related information on smartwatches taking into consideration changes in orientation and location of the device during different parts of the night.
	\item We extensively evaluate the performance of \systemname using measurements collected from two-week monitoring of $15$ participants. Our results demonstrate that \systemname can accurately capture a wide range of sleep events, estimate different sleep stages, and produce meaningful information about overall sleep quality.
\end{itemize}

%To summarize, the main contribution of this paper is the first smartwatch-based system that can capture a wider range of sleep events with a high
%accuracy. We show that, compared with existing mobile-based sleep monitoring solutions, the rich set of fine-grained sleep events given by
%o%
%ur approach can better capture a user's sleep patterns and quality across sleep stages.

\section{RESULTS}\label{sec:4experiment}
In this section, we detail the evaluation results for our system.

\subsection{Evaluation of Subcomponents}
We focus on the detection accuracy about five events, that are body posture, the body rollover, the hand position, the micro body movement and the acoustic events.
%, the classification of micro body movement

\subsubsection{Sleep Posture Classification Performance}
\label{subsub:bodyposture}

We test the overall classification performance of different body postures. The groundtruth of body postures are recorded by the cameras. We use a modified cross-validation approach by training the classifier with data from one user and testing the rest fourteen users. The motivation for using data from a single user as training data is to highlight the capability of \systemname to accurately characterize body posture with very little training data, while at the same time being able to generalize across users. The final performance is then calculated as the averaged accuracy across the 15 folds; as shown in Fig.~\ref{fig:posture_zhu}. We can observe that the posture detection accuracy is consistently high across all users, and does not show major variations across users. This good performance benefits from the distinct characteristics of arm position under different sleeping postures. Compared to results reported for SleepMonitor~\cite{sleepmonitor}, \systemname consistently improves performance which is mainly due to the template-based classifier that we use to verify classifications of the prone and supine states. In particular, \systemname achieves around $5$ percentage units higher performance on the prone state than SleepMonitor and overall has lower false positive rate. 

\begin{figure}
	\centering
	\begin{minipage}{.5\textwidth}
	 \centering
	\includegraphics[width=0.95\linewidth]{Figures/posture_zhu.pdf}
	\caption{Detection accuracy of body postures.}\label{fig:posture_zhu}
	\end{minipage}%
	\begin{minipage}{.5\textwidth}
			\centering
		\includegraphics[width=0.95\linewidth]{Figures/handposition_zhu.pdf}
		\caption{Identification accuracy of hand positions.}\label{fig:hand_zhu}
	\end{minipage}
\end{figure}

\begin{table}[!thbp]\footnotesize
%	\tabcolsep1pt
	\centering  % ������
	%\renewcommand\arraystretch{0.277}
	%\caption{The confusion matrix of body posture classification.}\label{tab:posture}
	%\noindent\makebox{%
	%\begin{tabular}{1\textwidth}{ c | c | c | c | c | c | c}
	\renewcommand\arraystretch{0.3}
	\caption{The confusion matrix of body posture classification.}\label{tab:posture}
	\begin{tabular}{c| c | c | c | c | c | c}
		\cline{1-7}
		&\multicolumn{1}{ c|}{ }
		& \multicolumn{4}{ c|}{ }\\
		\multirow{2}*{}
		&\multicolumn{1}{c|}{\multirow{2}*{{Result}}}
		&\multicolumn{4}{c|}{{Prediction}}
		& \multirow{4}*{{Recall}} \\
		%&\multicolumn{5}{ c |}{\textbf{\small Prediction}} \\
		% & \multicolumn{5}{ c |}{ } \\
		\cline{3-6}
		& & & & & \\
		\multicolumn{1}{c|}{{}}
		&  \multicolumn{1}{c|}{{}}
		&  \multicolumn{1}{c|}{{Supine}}
		&  \multicolumn{1}{c|}{{Left Lateral}}
		&  \multicolumn{1}{c|}{{Right Lateral}}
		&  \multicolumn{1}{c|}{{Prone}}   \\
		& & & & & \\
		\cline{1-7}
		& & & & & \\
		\multirow{5}{*}{\begin{sideways}{{Groundtruth}}\end{sideways}}
		&   {Supine}   & {\bf{{1182}}}    &   $25$      &   $4$      &   $9$    &   {96.7\%}\\
		& & & & & \\
		\cline{2-7}
		& & & & & \\
		&   {Left Lateral}   &   $6$      &   {\bf{{1292}}}     &   $0$      &   $0$   &   {99.5\%} \\
		& & & & & \\
		\cline{2-7}
		& & & & & \\
		&   {Right Lateral}   &   $7$      &   $0$      &  {\bf{{1275}}}      &   $12$  &   {98.5\%}  \\
		& & & & & \\
		\cline{2-7}
		& & & & & \\
		&   {Prone}   &   $19$      &   $2$      &   $3$      &   {\bf{{567}}}   &   {95.9\%} \\
		& & & & & \\
		\cline{1-7}
		& & & & & \\
		&   {Precision}    &   {97.3 \%}   &   {98.0\%}   &   {99.5\%}   &   {96.4\%}    \\
		& & & & & \\
		\cline{1-7}
	\end{tabular}
\end{table}

To have a deep evaluation about the sleep posture detection, we randomly choose one user to train the classifier. Then we calculate the detection precision and recall across postures. The result is shown in Table \ref{tab:posture}. The values in blocks are the corresponding numbers of four sleep postures from 14 test users. Due to that the angle features of acceleration are similar between the suspine posture with hand putting on the head and the left-lateral posture, a small amount of the supine postures are classified as left lateral.  The total amount of the prone posture is less than the number of other postures. It suggests that most people are not accustomed to sleep in the prone posture, because it is neither healthy nor comfortable. In coclude, Table \ref{tab:posture} shows the outstanding detection performance.



\subsubsection{Performance of body rollover counting}
To verify the efficiency of body rollover detection algorithm, we compare each user's  body rollover number detected by {\systemname} with the groundtruth recorded by camera. The performance is showed in Table \ref{tab:rollver}. We can see that User 3, User 4 and User 13 has an unusually high number. For User 3 and User 4, they have difficulty in falling asleep due to the sleep disorder.  User 13 needs to rollover frequently because of  his loundly snoring. For all the 15 users, the detection accuracies are all very high, and the least one is still 87\%. Thus {\systemname} can accurately distinguish the large hand movement from the body rollover in bed. What's more, detecting errors in body rollover events will not have a significant impact on our end result, because the division of  sleep stages is a comprehensive consideration of all the detected features in each stage, such as micro body movement and acoustic events.

\begin{table}[!thbp]\footnotesize
  %\centering  % ������
 % \tabcolsep 1pt
  %\arrayrulewidth1pt
  \caption{Detection accuracy of body rollover.}\label{tab:rollver}
   \renewcommand\arraystretch{1}{\multirowsetup}{\centering}
        \begin{tabular}{cccccccccccccccc}
        \toprule
         \textbf{User}    & 1& 2  & 3& 4& 5& 6& 7& 8& 9& 10& 11& 12& 13& 14& 15\\
        \midrule
         \rowcolor{Gray}      { Groundtruth}  &231&204&442&397&198&101&196&164&193&208&131&205&342&149&156 \\
                 { Accuracy} &91\%& 94\% &88\%&93\%&96\%&94\%&87\%&90\% &93\% &94\% &92\% &94\% &89\% &90\% &95\%\\
        \bottomrule
 \end{tabular}
\end{table}

\subsubsection{Performance of hand position recognition}
To test the recognition performance of different hand positions, {\systemname} uses the cross-validation approach presented in \ref{subsub:bodyposture} with only one user's data at a time to train the classifier and the remaining 14 users' data as the test sets. The classifier for detecting the hand movement trajectory is combined with the detection of periodic signals caused by respiration, then the hand position on the chest (or abdomen or head) can be identified. Fig. \ref{fig:hand_zhu} illustrates the accuracy of hand position across 15 users. As we can see that with just one set of training data, the accuracies for different users are all higher than 87\%. Therefore, our system can achieve a satisfied identification accuracy for different hand positions. Moreover, we find that at least four out of fifteen participants tend to put their hands on their heads; one participate unconsciously puts her hand on her chest which makes her have a nightmare sometimes. Those are all bad habits disrupting a good sleep.  {\systemname} can report such key findings to improve the users' sleep qualities.


\subsubsection{Performance of micro body movement detection}
To assess the detection accuracy of micro body movement, we manually label the ground truth  recorded by the camera during sleep, including hand moving, arm raising, and body trembling. We also use the accelerometer embedded in the smartphone which placed on the bed to record the occurrence of micro body movements, so as to avoid missing some movements such as trembling concealed by the quilt. Fig. \ref{fig:micro_movement_zhu} illustrates the detection accuracy across 15 users. It shows that the accuracies for all users are very close, that is, there will be no major changes between users. And from Fig. \ref{fig:micro_combine}, we find that even though the worst classification result belongs to the hand movement, the average precision value and recall value still exceed 75\%. The averaged accuracies of arm raising and body trembling are 93\% and 84\%, respectively. Because the training data volume for the hand movement and body trembling are small, so the performance can been improved by setting each user a threshold by collecting a longer term's sleeping data. In addition, the purpose of  micro body movement detection is to detect different sleep stages, and the hand movement usually appears in all sleep stages, thus the poor accuracy of hand moving does not have a significant impact on the final result. 


\begin{figure}
	\centering
	\begin{minipage}{.5\textwidth}
		 \includegraphics[width=0.85\linewidth]{Figures/micro_movement_zhu.pdf}
		\caption{Detection accuracy of micro body movement.}\label{fig:micro_movement_zhu}	
	\end{minipage}%
	\begin{minipage}{.5\textwidth}
	 \centering
	\includegraphics[width=0.85\linewidth]{Figures/micro_combine1.pdf}
	\caption{Performance of micro body movement detection.}\label{fig:micro_combine}
	\end{minipage}
\end{figure}



 \subsubsection{Performance of acoustic events detection}
To study the detection accuracies of different acoustic events, we compare the ground truth recorded by the camera with the detected results by our system. Table \ref{tab:sound} shows the results across 15 participants. We can see that the precision for cough event is 88.9\% and a little than other three events. The reason is that different user's cough patterns are different, the pre-defined parameters in the detection model does not include all possible patterns. In fact, we train the parameters with only 120 sets of nighttime sound data. Those data come from 40 (21 males and 19 females) volunteers of different ages (from 15 to 60 years old) who are prone to snoring, coughing, or somniloquy at night. To further improve the detection accuracy, we can train particular parameters for different users.

\begin{table}[!thbp]\footnotesize
% \tabcolsep1pt
  \centering  % ������
 \renewcommand\arraystretch{0.3}
  \caption{The confusion matrix of acoustic events detection.}\label{tab:sound}
\begin{tabular}{c| c | c | c | c | c | c}
   \hline
   &\multicolumn{1}{ c|}{ }
   & \multicolumn{4}{ c|}{ }\\
   \multirow{2}*{}
&\multicolumn{1}{c|}{\multirow{2}*{{ Result}}}
&\multicolumn{4}{c|}{{ Prediction}}
& \multirow{4}*{{ Recall}} \\
    %&\multicolumn{5}{ c |}{\textbf{\small Prediction}} \\
   % & \multicolumn{5}{ c |}{ } \\
    \cline{3-6}
    & & & & & \\
    \multicolumn{1}{c|}{{}}
    &  \multicolumn{1}{c|}{{}}
    &  \multicolumn{1}{c|}{{ Snore}}
    &  \multicolumn{1}{c|}{{ Cough}}
    &  \multicolumn{1}{c|}{{ Somniloquy}}
    &  \multicolumn{1}{c|}{{ Other}}   \\
    & & & & & \\
     \cline{1-7}
    & & & & & \\
    \multirow{5}{*}{\begin{sideways}{{ Groundtruth}}\end{sideways}}
    &   { Snore}   & {\bf{{96}}}    &   $0$      &   $0$      &   $9$    &   {91.4\%}\\
    & & & & & \\
    \cline{2-7}
    & & & & & \\
   &   { Cough}   &   $3$      &   {\bf{{64}}}     &   $0$      &   $4$   &   {90.1\%} \\
    & & & & & \\
     \cline{2-7}
    & & & & & \\
    &   { Somniloquy}   &   $0$      &   $3$      &  {\bf{{42}}}      &   $2$  &   {89.4\%}  \\
    & & & & & \\
     \cline{2-7}
    & & & & & \\
    &   { Other}   &   $0$      &   $5$      &   $4$      &   {\bf{{325}}}   &   {97.3\%} \\
    & & & & & \\
    \hline
    & & & & & \\
    &   { Precision}      &   {96.9\%}   &   {88.9\%}   &   {91.3\%}   &   {95.6\%}    \\
    & & & & & \\
    \hline
   \end{tabular}
\end{table}


\subsection{Overall performance}

\subsubsection{Performance of sleep stage detection}

In order to prove that the detected events not only reflect the user's sleep habits, but also effectively identify the sleep stages to assess the sleep quality, we regard the reported results from Fitbit Charge2 as the ground truth.  There are 50 sets of nocturnal sleep data selected from 210 sets of sleep data. To make the selection fairly, we randomly pick at least 3 sets of data for each participant. For  reflecting the sleep stage change carefully, {\systemname} evaluates the sleep stage every 15 minutes even when there are no events detected. The averaged precision value and recall value are shown in Table \ref{tab:sleep stage}. It indicates that though {\systemname} may make misjudgement between the light sleep and REM, the overall performance is satisfying.

\begin{table}[!thbp]\footnotesize
%	\tabcolsep1pt
	\centering  % ������
	\renewcommand\arraystretch{0.4}
	\caption{{The confusion matrix of sleep stage detection.}}\label{tab:sleep stage}
	\begin{tabular}{c| c | c | c | c | c}
		\hline
		&\multicolumn{1}{ c|}{ }
		& \multicolumn{3}{ c|}{ }\\
		\multirow{2}*{}
		&\multicolumn{1}{c|}{\multirow{2}*{{ Result}}}
		&\multicolumn{3}{c|}{{ Prediction}}
		& \multirow{3}*{{ Recall}} \\
		%&\multicolumn{5}{ c |}{\textbf{\small Prediction}} \\
		% & \multicolumn{5}{ c |}{ } \\
		\cline{3-5}
		& & & & & \\
		\multicolumn{1}{c|}{{}}
		&  \multicolumn{1}{c|}{{}}
		&  \multicolumn{1}{c|}{{ REM}}
		&  \multicolumn{1}{c|}{{ Light Sleep}}
		&  \multicolumn{1}{c|}{{ Deep Sleep}} \\
	%	& & & & & \\
		\cline{1-6}
		& & & & & \\
		\multirow{1}{*}{\begin{sideways}{{ Groundtruth}}\end{sideways}}
		&   { REM}   & {\bf{{476}}}    &   $143$      &   $61$     &   {70.0\%}\\
		& & & & & \\
		\cline{2-6}
		& & & & & \\
		&   { Light Sleep}   &   $131$      &   {\bf{{508}}}     &   $91$      &   {69.6\%} \\
		& & & & & \\
		\cline{2-6}
		& & & & & \\
		&   { Deep Sleep}   &   $63$      &   $113$      &  {\bf{{262}}}      &   {59.8\%}  \\
		& & & & & \\
		\cline{1-6}
		& & & & & \\
		&   { Precision}      &   {71.0\%}   &   {66.5\%}   &   {63.3\%}   \\
		& & & & & \\
		\hline
	\end{tabular}
\end{table}

\subsubsection{Effect of respiratory amplitude on sleep stage detection}
When we detect different sleep stages, we also consider the respiratory amplitude when the hand's position is in the abdomen or chest. To assess the effectiveness of respiration amplitude estimation, we evaluate the performance of the sleep stage detection in two cases, that are with and without taking the respiration amplitude into account. The performance of sleep stage detection are shown  in Table \ref{tab:respiratory}. For three different sleep stages, both the precision and recall values are improved with the help of respiration amplitude estimation.

\begin{table}[!thbp]\footnotesize
	\centering  % ������
	\renewcommand\arraystretch{0.3}
	\caption{Effect of respiration amplitude estimation.}\label{tab:respiratory}
	\begin{tabular}{c| c | c | c | c | c | c| c |}
		\cline{2-8}
		&\multicolumn{1}{ c|}{ }
		&\multicolumn{2}{ c|}{ }
		&\multicolumn{2}{ c|}{ }
		& \multicolumn{2}{ c|}{ }\\
		%  \multirow{4}*{}
		&\multicolumn{1}{c|}{}
		&\multicolumn{2}{c|}{\textbf{\footnotesize REM}}
		&\multicolumn{2}{c|}{\textbf{\footnotesize Light Sleep}}
		&\multicolumn{2}{c|}{\textbf{\footnotesize Deep Sleep}} \\
		%&\multicolumn{5}{ c |}{\textbf{\small Prediction}} \\
		% & \multicolumn{5}{ c |}{ } \\
		\cline{2-8}
		& & & & & & &\\
		\multicolumn{1}{c|}{\textbf{}}
		&  \multicolumn{1}{c|}{\textbf{Features}}
		&  \multicolumn{1}{c|}{\footnotesize Precision}
		&  \multicolumn{1}{c|}{\footnotesize Recall}
		&  \multicolumn{1}{c|}{\footnotesize Precision}
		&  \multicolumn{1}{c|}{\footnotesize Recall}
		&  \multicolumn{1}{c|}{\footnotesize Precision}
		&  \multicolumn{1}{c|}{\footnotesize Recall}\\
		& & & & & & &\\
		\cline{2-8}
		& & & & & & &\\
		\multirow{5}{*}
		&   \textbf{\footnotesize Without Respiration Amplitude}   & $62.9\%$    &   $63.4\%$      &   $59.4\%$      &   $63.9\%$    &   $57.7\%$ &  $54.1\%$ \\
		& & & & & & &\\
		\cline{2-8}
		& & & & & & &\\
		&   \textbf{\footnotesize With Respiration Amplitude}   &   $71.0\%$      &   $70.0\%$     &   $66.5\%$      &   $69.7\%$   &   $63.3\%$ &   $59.8\%$ \\
		& & & & & & &\\
		
		\cline{2-8}
		
	\end{tabular}
\end{table}

\subsubsection{Performance comparison}
We compare {\systemname} with two state-of-the-art work, the sleep detection app Sleep As Android and smartphone-based system Sleep Hunter \cite{gu2016sleep}.  Considering that Sleep As Android can only detect light sleep stage and deep sleep stage, we only compare the performance of these two stages. Table \ref{tab:comparison} shows the detection results.
As we can see, {\systemname}  performs much better than Sleep As Android and slightly better than Sleep Hunter. This good performance comes from the incorporation of rich and complicated sleep events. Therefore, our system helps the users understand their sleep more easily and improve  sleep quality more effectively.

  \begin{table}[!thbp]\footnotesize
 	\centering  % ������
 	\renewcommand\arraystretch{0.3}
 	\caption{Performance of sleep stage detection comparison.}\label{tab:comparison}
 	\begin{tabular}{c| c | c | c | c | c |}
 		\cline{2-6}
 		&\multicolumn{1}{ c|}{ }
 		&\multicolumn{2}{ c|}{ }
 		&\multicolumn{2}{ c|}{ }\\
 		%  \multirow{4}*{}
 		&\multicolumn{1}{c|}{}
 		&\multicolumn{2}{c|}{\textbf{\footnotesize Light Sleep}}
 		&\multicolumn{2}{c|}{\textbf{\footnotesize Deep Sleep}} \\
 		%&\multicolumn{5}{ c |}{\textbf{\small Prediction}} \\
 		% & \multicolumn{5}{ c |}{ } \\
 		\cline{2-6}
 		\multicolumn{1}{c|}{\textbf{}}
 		&  \multicolumn{1}{c|}{\diagbox{System}{Stage}}
 		&  \multicolumn{1}{c|}{\footnotesize Precision}
 		&  \multicolumn{1}{c|}{\footnotesize Recall}
 		&  \multicolumn{1}{c|}{\footnotesize Precision}
 		&  \multicolumn{1}{c|}{\footnotesize Recall}\\
 		\cline{2-6}
 		& & & & & \\
 	%	\multirow{3}{*}
 		&   \textbf{\footnotesize SleepGuard}   & $66.5\%$    &   $69.6\%$      &   $63.3\%$      &   $59.8\%$  \\
 		& & & & &  \\
 		\cline{2-6}
 		& & & & & \\
 		&   \textbf{\footnotesize Sleep As Android}   &   $27.8\%$      &   $35.4\%$     &   $35.7\%$      &   $50.2\%$   \\
 		& & & & &  \\
 		\cline{2-6}
 		& & & & & \\   
 		&   \textbf{\footnotesize Sleep Hunter}   &   $66.74\%$      &   $66.11\%$     &   $60.00\%$      &   $50.73\%$   \\
 		& & & & &  \\
 		
 		\cline{2-6}
 		
 	\end{tabular}
 \end{table}


Further, we compare {\systemname}'s functions with 8 other sleep detection products  in Table \ref{tab:function}, including Sleep As Android, Sleep Hunter, sleepMonitor \cite{sleepmonitor}, Sleeptracker \cite{sleeptracker}, Fitbit, isleep \cite{hao2013isleep}, Jawbone \cite{Jawbone} and ubiSleep \cite{pombo2016ubisleep}.  {\systemname} detects a wider range of sleep events and thus provides a better user experience.

\begin{table*}[!thbp]\footnotesize
%\setlength{\abovecaptionskip}{0.8pt}
  \centering  % ������
  \tabcolsep7pt
  %\arrayrulewidth1pt
  \caption{Functions comparision between different systems.}\label{tab:function}
  \renewcommand{\multirowsetup}{\centering}
  \noindent\makebox[\textwidth]{%
        \begin{tabularx}{1.0\textwidth}{|c|c|c|c|c|c|c|}
        \cline{1-7}
        \multicolumn{1}{|c|}{\multirow{2}*{\textbf{\footnotesize System}}}
        &\multicolumn{6}{c|}{\textbf{ \footnotesize Detected Events}} \\
         \cline{2-7}
    &  \multicolumn{1}{c|}{\textbf{ \footnotesize Heart Rate }}
    &  \multicolumn{1}{c|}{\textbf{ \footnotesize Acoustic Event }}
    &  \multicolumn{1}{c|}{\textbf{ \footnotesize Sleep Posture }}
     &  \multicolumn{1}{c|}{\textbf{ \footnotesize Body Movement }}
      &  \multicolumn{1}{c|}{\textbf{ \footnotesize Hand Position}}
       &  \multicolumn{1}{c|}{\textbf{ \footnotesize Sleep Stage}} \\
        \cline{1-7}
        \multirow{7}{2.14cm}
        {\textbf{SleepGuard\\Sleep as Android\\Sleep Hunter\\SleepMonitor\\Sleeptracker\\isleep \\Fitbit\\Jawbone\\ ubiSleep } }
        & &$\checkmark$ & $\checkmark$ &  $\checkmark$  &$\checkmark$ &$\checkmark$\\
        & &$\checkmark$ & & & &$\checkmark$\\
        & & $\checkmark$& &$\checkmark$ & &$\checkmark$\\
        & & & $\checkmark$ & & &\\
        &$\checkmark$ & & & & &$\checkmark$\\
         & &$\checkmark$ &   &$\checkmark$ & &\\
         &$\checkmark$ & & & & &$\checkmark$ \\
        & & & & & &$\checkmark$ \\
        & $\checkmark$&$\checkmark$ & & & &\\
        \cline{1-7}
 \end{tabularx}}
\end{table*}

\subsubsection{User survey}
We conducted a user survey of 15 volunteers participating in our experiments to evaluate the user experience. In order to get their subjective feeling of sleep quality and use experience for our system, we asked participants to fill out our questionnaire based on PSQI \cite{carpenter1998psychometric} every morning during the experiments. The questions in our survey include:
\begin{enumerate}
  \item Subjective sleep quality (5 levels, 1 for excellent and 5 for worst),
  \item Sleep duration,
  \item Sleep disturbances,
  \item Daytime dysfunction.
\end{enumerate}
For the above four items, each one is rated on a 1 to 5 scale. These scores are first summed to yield a total score, which ranges from 0 to 20. Then we merge every five neighboring scores into one scale and eventually divide the total scores into four levels, recorded as 0, 1, 2 and 3, representing poor, general, good and excellent, respectively. The final sleep quality score comes from the comprehensive scores of above four questionnaires. 

The 14-days sleep quality scores from 15 participates are presented in Table. \ref{tab:quality}. We also list the sleep quality assessments by our system and Fitbit.  We can see the result is satisfactory and representative to some extent.
	
In addition, we also ask whether users are interested in these events, such as sleep posture, hand position, which detected by {\systemname}. 80\% of participants believe that the detection of sleep posture is very necessary, showing their sleep posture can not only help people to avoid health problems caused by long-term improper sleeping posture, but also help us find out the reasons for the next day's physical discomfort, such as dizziness, muscle soreness may be due to improper sleeping position. 60\% of the participants thought it useful to detect their hand position in the supine position and even one user mentioned that he did often have nightmares and that he found his hands were often placed on his chest. And {\systemname} was able to remind him to avoid such a hand position when sleeping. However, only 20\% of them think it is necessary to calculate the number of body rollover, but the detection of body rollover is very useful in the division of sleep stage.


\begin{table} \footnotesize
  \centering  % ������
  \renewcommand\arraystretch{0.5}
  \caption{\textcolor{blue}{Results of sleep quality assessment.}}\label{tab:quality}
\begin{tabular}{c| c | c | c | c | c |c |c |c |c |c| c |c |c |c |c |c |c|}
   %\cline{2-6}
   %&\multicolumn{1}{ c|}{ }
   %&\multicolumn{2}{ c|}{ }
  % &\multicolumn{2}{ c|}{ }\
    %&\multicolumn{1}{c|}{}
   %&\multicolumn{2}{c|}{\textbf{\scriptsize Light Sleep}}
  % &\multicolumn{2}{c|}{\textbf{\scriptsize Deep Sleep}} \\
    %&\multicolumn{5}{ c |}{\textbf{\small Prediction}} \\
   % & \multicolumn{5}{ c |}{ } \\
    \cline{2-17}
    \multicolumn{1}{c|}{\textbf{}}
    &  \multicolumn{1}{c|}{\diagbox{System}{User ID}}
    &  \multicolumn{1}{c|}{\footnotesize  1}
    &  \multicolumn{1}{c|}{\footnotesize  2}
    &  \multicolumn{1}{c|}{\footnotesize  3}
    &  \multicolumn{1}{c|}{\footnotesize  4}
    &  \multicolumn{1}{c|}{\footnotesize  5}
    &  \multicolumn{1}{c|}{\footnotesize  6}
    &  \multicolumn{1}{c|}{\footnotesize  7}
    &  \multicolumn{1}{c|}{\footnotesize  8}
    &  \multicolumn{1}{c|}{\footnotesize  9}
    &  \multicolumn{1}{c|}{\footnotesize  10}
    &  \multicolumn{1}{c|}{\footnotesize  11}
    &  \multicolumn{1}{c|}{\footnotesize  12}
    &  \multicolumn{1}{c|}{\footnotesize  13}
    &  \multicolumn{1}{c|}{\footnotesize  14}
    &  \multicolumn{1}{c|}{\footnotesize  15}\\
     \cline{2-17}
    & & & & & & & & & & & & & & & & \\
    \multirow{16}{*}
    &   \textbf{\footnotesize SleepGuard}  & $3$ & $3$ & $0$ & $2$ & $2$ & $1$ & $3$ & $0$ & $2$ & $2$ & $2$ & $2$ & $1$ & $0$ & $2$ \\
   & & & & & & & & & & & & & & & &\\
    \cline{2-17}
   & & & & & & & & & & & & & & & &\\
   &   \textbf{\footnotesize Fitbit}   & $3$ & $3$ & $0$ & $3$ & $1$ & $0$ & $2$ & $3$ & $2$ & $2$ & $2$ & $2$ & $2$ & $1$ & $2$\\
      & & & & & & & & & & & & & & & & \\
      \cline{2-17}
       & & & & & & & & & & & & & & & & \\
    &   \textbf{\footnotesize User survey}  & $3$ & $2$ & $0$ & $2$ & $2$ & $0$ & $3$ & $1$ & $1$ & $2$ & $2$ & $3$ & $0$ & $0$ & $2$ &  \\
     & & & & & & & & & & & & & & & & \\
    \cline{2-17}
   \end{tabular}
\end{table}

%\begin{figure}
 %\centering
% \includegraphics[width=0.52\linewidth]{Figures/quality.pdf}
 %\caption{Participants' sleep quality distribution}\label{fig:quality}
%\end{figure}

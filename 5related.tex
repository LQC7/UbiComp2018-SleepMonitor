\section{RELATED WORK}\label{sec:5related}

Sleep quality is a crucial issue for each individual, and the poor sleep would lead to numerous diseases, such as  endocrine dyscrasia, depression, immunity decline \cite{vgontzas2009insomnia,gottlieb2005association}. Thus, a lot of research works have been proposed to monitor the sleep \cite{langkvist2012sleep,hao2013isleep,bai2012will,kay2012lullaby,bain2003evaluation,pombo2016ubisleep}. Below, we summarize the related state-of-the-art research works as the following three categories.

\textbf{Medical technology based work}. Traditionally, the dedicated medical technologies, like EEG, ECG and EMG \cite{saper2005hypothalamic}, have been applied for sleep monitoring. Those technologies rely on the certain biomedical signals, such as brain wave, muscle tone and eye movement, to assess the sleep quality. For example, the EEG technology in \cite{langkvist2012sleep,oropesa1999sleep,ebrahimi2008automatic} monitors the brain waves, and then recognizes the sleep stages by leveraging unsupervised learning approaches.

Although a high accuracy achieved by those  technologies, they have two drawbacks. First, those  technologies require the dedicated medical devices, which are very expensive compared the widely available  smart watch/phone. Second, they require the users to be attached lots of sensors on the human body, which may cause a healthy person had to sleep and result in large errors.

Compared with those medical technologies, our system has two advantages. First, we only need a smartwatch, which is cost effective. Second, the smartwatch does not disturb a user's normal sleep, thus we can monitor the user's sleep quality more accurately.

\textbf{Smartphone based work}. Recently, some researches use the smartphones for sleep monitoring \cite{hao2013isleep,bai2012will,kay2012lullaby,choe2011opportunities} . iSleep \cite{hao2013isleep} measures the sleep quality by recording the sleep-related acoustic events and evaluates the sleep quality using the Pittsburgh Sleep Quality Index (PSQI) \cite{carpenter1998psychometric}. Bai et al. \cite{bai2012will} predicts a user's sleep quality by observing the user's daily activities with the smartphone's sensor data, like the accelerometer, gyroscope, microphone, etc. Work in \cite{kay2012lullaby} leverages the smartphone sensors to record the sleep disruptors for a user. The authors in \cite{choe2011opportunities}  explore a series of opportunities to support healthy sleep behaviors based on the smartphone sensors. Several research works predict the sleep quality by using the smartphone to monitor the external  influence  factors, such as the daily activity, sleeping environment, location and family settings \cite{chen2013unobtrusive,zhang2013real}. Besides those research works, many Smartphone Apps, such as Sleep As Android \cite{SleepAndroid}, Sleep Journal \cite{SleepJournal}, YawnLog \cite{YawnLog} also have been developed to monitor a uer's sleep quality.

Those smartphone based systems, however, require placing the smartphone at a specifically location near to the user, which usually cannot be satisfied in reality. For example, Gu et al. \cite{gu2016sleep} needs the smartphone to be placed next to the user's head, and requires the smartphone keeping stationary throughout the sleeping process. But, many users do not want to place the mobile phone too close to the body due to health risk concerns  \cite{StepHealth,Quorasleep}.

Compared with the existing smartphone based systems, our system uses the commodity smartwatch for sleep monitoring. Since many users are willing to wear the smartwatch throughout the night, thus the smartwatch can remain relatively close to the user over the duration of sleep. The smartwatch also allows us to collect a richer set of information, which enables a wider range of sleep-related events and more accurate to achieve sleep detection and sleep quality assessment.


\textbf{Smartwatch based work}. With the widely use of wearable devices, more and more researchers and industries try to  use the smartwatch or wearable-wrist for sleep monitoring \cite{bain2003evaluation,bonnet2003insomnia,pombo2016ubisleep,caviness1996myoclonus}.  The Sleeptracker
\cite{sleeptracker} is a wristwatch-shaped unit that apart from telling the time, also infers whether the user is in deep sleep, light sleep, or awake, using an accelerometer. The ubiSleep \cite{pombo2016ubisleep} joints heart rate, accelerometer, and sound signals collected into the smartwatch for noninvasive sleep monitoring. The aXbo alarm clock [9] is packaged as a stand-alone application in the form of an alarm clock that wirelessly communicates with a wrist-band unit. The Zeo \cite{caviness1996myoclonus} is similarly using an alarm clock base unit with a worn sensing device, but the latter is a headband rather than a wrist-band that based on electroencephalograph (EEG) to monitor sleep. We know Zeo has good performance in sleep stage detection compared to some wristband sleep monitoring products because products based on some biological signals like EEG are able to get more accurate and more representative sleep data than actigraphy-based \cite{Actigraphy} sleep monitoring products. However, the vast majority of biosignal-based sleep detection approaches require specialized equipment, which is high cost and complex to operate, while the approaches based on actigraphy are well adapted and user-friendly to accept and understand.

Moreover, these actigraphy-based wristband devices or smartwatch Apps only can gather coarse-grained information and  do not design a model for deep understanding the relationship between a user��s sleep pattern and the sensor data. For example, many smartwatch Apps, like Jawbone Up \cite{Jawbone}, FitBit \cite{fitbit}, YawnLog \cite{YawnLog} and WakeMate \cite{WakeMate}, do not show how they assess a user��s sleep quality based on what kind of sensor data.

Compared with the existing smartwatch or wearable-wrist based systems, our system can collect an extensive set of sleep-related events, many of which are not supported in prior work. Our system enables users to gain a deeper understanding of their sleep patterns and the causes of poor sleep.

\section{MOTIVATION}\label{Sec:6motivation}
In this section, we will explain our system motivation.
Sleep takes most of the time in a person's life, and a good sleep plays a very important role throughout the day. Sleep can help people restore the body's function, store energy, and repair damaged cells. According to medical reports, good sleep can make the body and mind get appropriate rest and conditioning, promote metabolism simultaneously. If the lack of sleep time or poor sleep environment for a long time will cause mental malaise, inattention, and even life-threatening. So how to create high sleep quality is essential and necessary. Therefore, to improve the quality of sleep, first of all, we should clarify the factors that affect the quality of sleep. Then, we find a standard to assess the quality of sleep to help people improve sleep.

According to our survey, we can classify the factors that affect sleep quality into the following, which is why our system selects some of these events for detection.
\begin{itemize}
  \item Personal sleep habits: sleep habits include pre-sleep diet, sleeping posture, the hand position and so on. For diet, we should try to avoid eating sugary and fatty foods too much, it will change your blood sugar level, and then delay sleep \cite{eating2013,eating2014}. And sleeping posture should be paid more attention by us. According to research, dreaming and sleep quality are associated with underlying brain functions and may be affected by body posture \cite{posture2004}. It has been well stated that right-side sleepers had better subjective sleep quality than left-side sleepers. And for different people, they should have his own fit sleeping posture, taking into account his own physical factors \cite{posture2016,posture2017}. For example, people with ailments such as heart disease or high blood pressure are unfit for prone position. We can consciously avoid or take some kind of sleep posture, which will have very good benefits for your sleep quality and health \cite{posture2015}. Further, if a proper hand position isn't regulated and maintained, it can cause a massive array of problems for sleep \cite{position2014}. For example, if you get used to putting your hand on your head when you sleep, you will feel tingling and numbness. even after moving position, an underlying symptom may be the early stages of carpal tunnel syndrome. So in {\systemname}, we incorporate sleep position and hand position into the detection of events so that the factors that affect the quality of sleep can be more widely considered.
  \item sleep environment: the bedroom environment can have a significant influence on sleep. Both the light, noise, and temperature can be the culprit in poor sleep. We have found that too much light exposure can shift our biological clock , which makes restful sleep difficult to achieve, it affects our sleep and wake cycle \cite{light2007}. Besides, we also note that the dim light will affect our sleep too. According to the study \cite{light2016}, it can be learned that the dim artificial light during sleep is significantly associated with the general increase in fatigue. So the proper light can be used to increase the sense of exhaustion, thereby promoting sleep. As for noise, in order to promote better sleep, we need to reduce the volume of noise as much as possible. And the ideal temperature range for sleeping, there is no prescribed best standard among different individuals \cite{light2007}. So in {\systemname}, we mainly detect the illumination condition in the environment and make it as one of evaluation indicators for sleep quality.
  \item Physiological factors: physiological factors will have an impact on the structure and distribution of sleep. When the body is not comfort, such as cough, body aches, etc, the human cerebral cortex cells are always in an excited state, can not enter the inhibition and limit the depth of sleep and allow only brief episodes of sleep between awakenings, it like many other sleep disruptions. And long-term snoring can also have a serious impact on health and sleep. It can cause behind sleep apnea or narcolepsy, a sleeping disorder \cite{snoring2016,snoring2013}. However, Snoring is subject to many influences such as body position, sleep stage and the presence or absence of sleep-disordered breathing \cite{snoring2010}. So our system have taken the acoustic events during sleep into consideration, including snore,cough and somniloquy.
  \item Psychological factors: psychological factors are also one of the important reasons that affect the quality of sleep. During the daytime, people are mainly awakened by their daily activities. This is awakening. When people sleep at night, they mainly show up as a sleeping substance, which results in depression and sleep. If the excitement and inhibition can not maintain a certain balance to complete the awakening and sleep conversion, it will have insomnia and other symptoms. what's more, individuals of all ages who experience stress, anxiety, and depression tend to find it more difficult to fall asleep \cite{mood2003,light2007,mood2015}. For this factor, we can use the way of asking questions to record the user's psychological state before sleep
\end{itemize}

In addition to these events, we also considered the body rollover event during sleep, as turning over is also a necessary exercise during sleep \cite{rollover2014,rollover201408}. We can get some reflections of sleep status by counting the number of body rollover during a whole night. And we can understand the user's detailed sleep information by detecting  micro body movement (hand moving, arm raising, and body trembling) during sleep.

According to our investigation, The quality of sleep is largely determined by the percentage of different sleep stages throughout the sleep \cite{iSleep,gu2016sleep}, so we can segment the user's entire sleep process into sleep stages and calculate the proportion of each sleeping stage in the whole sleep process, in order to assess the quality of sleep.  Sleep stage refers to the three main stages, namely Rapid Eye Movement (also called dream stage), light sleep stage and deep sleep stage. The light sleep and deep sleep are also called non-Rapid Eye Movement (NREM). And the division of the sleep stage is based on the fact that sleepers usually exhibit distinguishable physical activities in different sleep stages \cite{ancoli2003role}. For example, body trembling and somniloquy often occur in REM while large body movements and snoring are more prone to occur in the deep sleep stage, and the roll-over frequency can also help us to classify the sleep stage \cite{rollover2007}. So we divide the sleep stage based on detecting these specific events  that belong to different stages of sleep. In addition, the quality of sleep is also affected by age, sleep mate and other factors, we will further consider in future work.


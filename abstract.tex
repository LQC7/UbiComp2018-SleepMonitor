<<<<<<< HEAD
\begin{abstract}
Sleep is an important part of our daily routine -- we spend about one-third of our time doing it. By tracking the sleep-related events
and activities, sleep monitoring provides the decision support to help us understand the sleep quality and the causes of poor sleep.
Wearable devices provide a new way for sleep monitoring, allowing us to monitor sleep from the comfort of our own home, but existing
solutions do not take full advantage of the rich sensor data provided by these portable devices. In this paper, we develop a novel
approach to track a wide range of sleep-related events using smartwatches. We show that it is possible to track, using a single
smartwatch, sleep events like body postures and movements, acoustic events, and illumination conditions. From these events, a statistical
model can be designed to evaluate the user's sleep quality across various sleep stages. We evaluate our approach by conducting extensive
experiments involved ten users across a 2-week period. Our experimental results show that our approach can track a richer set of sleep
events, which provides better decision support for evaluating the sleep quality when compared to prior work.
\end{abstract}
=======
\begin{abstract}
Sleep is an important part of our daily routine -- we spend about one-third of our time doing it. By tracking the sleep-related events
and activities, sleep monitoring provides the decision support to help us understand the sleep quality and the causes of poor sleep.
Wearable devices provide a new way for sleep monitoring, allowing us to monitor sleep from the comfort of our own home, but existing
solutions do not take full advantage of the rich sensor data provided by these portable devices. In this paper, we develop a novel
approach to track a wide range of sleep-related events using smartwatches. We show that it is possible to track, using a single
smartwatch, sleep events like body postures and movements, acoustic events, and illumination conditions. From these events, a statistical
model can be designed to effectively evaluate a user's sleep quality across various sleep stages. We evaluate our approach by conducting
extensive experiments involved ten users across a 2-week period. Our experimental results show that our approach can track a richer set
of sleep events, which provides better decision support for evaluating the sleep quality, when compared to prior work.
\end{abstract}
>>>>>>> 8ff952559e42dc8fe674cbb4b2feae3334fa8483

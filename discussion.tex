\section{DISCUSSION}\label{sec:discussion}
In this work, we try to examine the possibilities of using smart wearables to detect sleep. Our focus is to provide a series of ways to detect sleep-related events that will greatly assist in the assessment of sleep quality.While our results are satisfactory, there are still a handful of issues to address as we highlight below.

\begin{itemize}
  \item \textbf{The accuracy of sleep stage detection.}
  Compared with polysomnography-based sleep stage detection accuracy, {\systemname} detection accuracy is difficult to achieve such an effect. This is because polysomnography monitors and analyzes sleep based on information such as EEG, EMG, EOG, and oxygen saturation, while {\systemname} only combine the body movement, acoustic events, sleep environment and other events during sleep to predict the sleep stage. Therefore, {\systemname} can not replace the professional medical equipment for high-precision sleep detection. However, SleepGuard is able to take advantage of the popularity of business smartwatches to provide easy-to-use and less intrusive sleep detection services, making it easy for users to learn about their sleep and make adjustments based on suggestions.
  \item \textbf{Battery life.}
  A major requirement for sleep detection is for the device to last long enough to continuously monitor a sleeper. In other words, battery life is an important requirement and the battery consumption directly impacts its applicability. As we know that battery capacity on smartwatch is very limited, so we need to try our best to reduce the resource consumption on smartwatch. For our work, we only use smartwatch as devices for data collection and delivery, while other computing capabilities are mainly implemented on smartphone, and we also have adopted simpler algorithms to reduce resource usage. But despite this, our smartwatch can only continue to collect about 6 hours of sensor data, which is not sufficient for longer sleep detection, so we still need to further study how to reduce power consumption. One possible solution is to change the data collection strategy to dynamically adjust the sampling frequency based on whether the body is moving or not.
  \item \textbf{Sensor data.}
  One limitation of {\systemname} is that heart rate data have not been taken into consideration by us when collecting sleep related data, which could result in some inaccuracies in the detection of the sleep stage. This is because currently we still can not get the original data collected by the heart rate sensor in the smartwatch and analyze the data. We can only obtain the output from the heart rate monitoring application that is included in the smartwatch. Such data is not accurate enough and does not make sense for us to analyze it.To compensate for the lack of heart rate data on the impact of system performance,  we have added an additional consideration of the respiratory amplitude, which can help us improve the performance of the sleep stage detection (it has been proved in our experiment).
  \item \textbf{Limitations of the algorithm.}
  When we try to detect sleep posture, we find a corresponding relationship between the position of the arm and the sleeping posture, so we indicate the sleeping posture by detecting its position. But currently we consider some specific positions of the arm in the four sleeping positions, some unusual arm's positions are ignored by us and that is something we need to improve further in future work.
\end{itemize}
